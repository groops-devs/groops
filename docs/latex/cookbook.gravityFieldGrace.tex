\section{GRACE gravity field recovery}\label{cookbook.gravityFieldGrace}
This cookbook chapter describes an example of estimating a gravity field solution using GRACE observation
data. For the respective month a set of spherical harmonic coefficients up to a maximum degree is determined.
An example scenario for this task can be found at
\url{https://ftp.tugraz.at/outgoing/ITSG/groops/scenario/scenarioGraceGravityfieldRecovery.zip}
including the required GROOPS scripts and data sets for the gravity field recovery process.
The background models are provided at \url{https://ftp.tugraz.at/outgoing/ITSG/groops/data/}.


\subsection{Background models}\label{cookbook.gravityFieldGrace:backgroundModels}
The following background models were used during the data processing:
\begin{itemize}
  \item \textbf{Earth rotation}: \href{https://www.iers.org/IERS/EN/Publications/TechnicalNotes/tn36.html}{IERS 2010} % IERS 2010
  \item \textbf{Moon, sun and planets ephemerides}: \href{https://ipnpr.jpl.nasa.gov/progress_report/42-196/196C.pdf}{JPL DE432} %JPL DE432
  \item \textbf{Earth tide}: \href{https://www.iers.org/IERS/EN/Publications/TechnicalNotes/tn36.html}{IERS 2010} % IERS 2010
  \item \textbf{Ocean tide}: \href{https://www.aviso.altimetry.fr/es/data/products/auxiliary-products/global-tide-fes/description-fes2014.html}{FES2014b} % FES2014b
  \item \textbf{Pole tide}: \href{https://www.iers.org/IERS/EN/Publications/TechnicalNotes/tn36.html}{IERS 2010} % IERS 2010
  \item \textbf{Ocean pole tide}: \href{https://doi.org/10.1029/2001JC001224}{Desai 2004} % Desai 2004 (IERS 2010)
  \item \textbf{Atmospheric tides}: \href{https://doi.org/10.1093/gji/ggx302}{AOD1B RL06} % AOD1B RL06
  \item \textbf{Atmosphere and Ocean Dealiasing}: \href{https://doi.org/10.1093/gji/ggx302}{AOD1B RL06} % AOD1B RL06
  \item \textbf{Sub-monthly continental hydrology}: \href{https://doi.org/10.2312/GFZ.b103-08095}{LSDM (ESMGFZ)} % LSDM (ESMGFZ)
  \item \textbf{Relativistic corrections}: \href{https://www.iers.org/IERS/EN/Publications/TechnicalNotes/tn36.html}{IERS 2010} % IERS 2010
\end{itemize}
These models were reduced during the analysis process and are not present in the solution.
The \href{https://doi.org/10.5194/essd-13-99-2021}{GOCO06s}
model was used as the static gravity field as well as for the trend component and annual oscillation.
In the script \verb|000groopsBackgroundModels.xml| a monthly mean of the GOCO06s including the time-variable components is determined
in form of time splines using \program{Gravityfield2TimeSplines}. This model is later added back to the final gravity solution.

\subsection{Instrument data preparation}\label{cookbook.gravityFieldGrace:metadata}
The ITSG gravity field solutions are computed from the official GRACE L1B \href{https://doi.org/10.5067/GRJPL-L1B03}{JPL (2018)}
and GRACE-FO L1B \href{https://doi.org/10.5067/GFL1B-ASJ04}{JPL (2019)} observation data. The data sets for this example are provided
in GROOPS file format in the scenario folder.

The satellite-to-satellite-tracking (SST) data consists of:
\begin{itemize}
  \item \textbf{K-band range rates}
  \item \textbf{Light time correction}
  \item \textbf{Antenna offset corrections}
\end{itemize}

Additional observation data required for the processing comprises:
\begin{itemize}
  \item \textbf{Star camera observations}
  \item \textbf{Accelerometer data}
  \item \textbf{Approximate orbits} % \textbf{Initial dynamic orbits}
  \item \textbf{Thruster data}
\end{itemize}

The determination of
\begin{itemize}
  \item \textbf{Kinematic orbits}
  \item \textbf{3x3 epoch covariances}
\end{itemize}
is depicted in \reference{Kinematic orbit determination of LEO satellites}{cookbook.kinematicOrbit}.
These data sets are also provided in the scenario folder.

Data preparation is handled in the script \verb|010groopsInstruments.xml|. The approximate orbits (initial dynamic orbits)
of the satellites, the star camera observations, the accelerometer data and the thruster data are resampled with a 5s sampling
and small gaps in the data are filled using \program{InstrumentResample}. Gross outliers are removed using \program{InstrumentRemoveEpochsByCriteria}
and the data is synchronized using  \program{InstrumentSynchronize}.

The approximate orbits are later used as a priori information for the dynamic orbit integration.
In addition to the observed orientation of the spacecrafts (star camera observations),
the nominal orientation is computed using \program{SimulateStarCameraGrace}. The difference between
observed and simulated orientation is determined using \program{InstrumentStarCameraMultiply}
and is employed in the outlier detection.

The accelerometer data is initially calibrated by estimating a bias using
\program{InstrumentAccelerometerEstimateBiasScale} with respect to simulated data created
with \program{SimulateAccelerometer}. For simulating accelerometer data a satellite model
implying the satellite's mass and surfaces is required. Such a model can be created with
\program{SatelliteModelCreate}. Models for the GRACE and GRACE-FO satellites are also provided
at \url{https://ftp.tugraz.at/outgoing/ITSG/groops/data/satelliteModel/}. Non-gravitational forces
comprising atmospheric drag, solar radiation pressure and albedo have to modeled when simulating
the accelerometer data. The acceleration bias parameters are determined as degree 3 time splines
with 6h nodes. When determining these parameters the thruster events are excluded from the estimation.

The SST observations, the light time corrections and the antenna center corrections are synchronized
with a 5s sampling together with simulated SST data created with \program{SimulateSatelliteTracking}.
Simulated data is used for the outlier detection of the original SST observations.

The sampling of the kinematic orbits is reduced to 60s using \program{InstrumentReduceSampling} and
an outlier detection is performed using the approximate dynamic orbits.

The approximate orbits, the star camera observations and the accelerometer data are divided into 24h arcs
(variational arcs). The kinematic orbits, its 3x3 epoch covariances, KBR observations,
light time corrections, antenna center corrections and star camera observations are divided into 3h arcs
per day (short arcs). Additionally the approximate orbits and the star camera observations are also
synchronized to short arcs.

Further information on instrument data preparation can be found in
\reference{Instrument data handling}{cookbook.instrument}.

\subsection{Variational equations}\label{cookbook.gravityFieldGrace:variational}
In this processing step dynamic orbits are computed for a complete 24h orbit arc by integrating
the forces acting on the GRACE/GRACE-FO satellites. Additionally, the state transition matrix is set up.
The dynamic orbits are then fitted to kinematic orbits and SST observations in a least squares adjustment
by co-estimating additional accelerometer calibration parameters together with the initial state vector.
The newly estimated parameters are then used to re-estimate the dynamic orbits and setting up the new
state transition matrix.

The script \verb|020groopsVariational.xml| in the scenario folder implements the required processing steps.
Time splines from a time-variable gravity field are estimated using \program{Gravityfield2TimeSplines}.
In this step the static gravity field (GOCO06s) is combined with the following time-variable components:
\begin{itemize}
    \item \configClass{gravityfield:potentialCoefficients}{gravityfieldType:potentialCoefficients}:
    static gravity field
    \item \configClass{gravityfield:trend}{gravityfieldType:trend}
    \begin{itemize}
        \item \configClass{gravityfield:potentialCoefficients}{gravityfieldType:potentialCoefficients}:
      trend component of gravity field
    \end{itemize}
    \item \configClass{gravityfield:oscillation}{gravityfieldType:oscillation}
    \begin{itemize}
        \item \configClass{gravityfieldCos:potentialCoefficients}{gravityfieldType:potentialCoefficients}:
      annual cosine component of gravity field
        \item \configClass{gravityfieldSin:potentialCoefficients}{gravityfieldType:potentialCoefficients}:
      annual sine component of gravity field
    \end{itemize}
    \item \configClass{gravityfield:timeSplines}{gravityfieldType:timeSplines}: atmosphere and ocean dealiasing (AOD1B RL06)
    \item \configClass{gravityfield:timeSplines}{gravityfieldType:timeSplines}: hydrology dealiasing (LSDM)
    \item \configClass{doodsonHarmonicTide}{tidesType:doodsonHarmonicTide}: ocean tides (FES2014b)
    \item \configClass{doodsonHarmonicTide}{tidesType:doodsonHarmonicTide}: atmospheric tides (AOD1B RL06)
    \item \configClass{tides:poleTide}{tidesType:poleTide}: pole tides (IERS 2010)
    \item \configClass{tides:poleOceanTide}{tidesType:oceanPoleTide}: ocean pole tides (IERS 2010)
\end{itemize}
\config{maxDegree}=\verb|220| and \config{sampling}=\verb|10/1440| is sufficient.

In \program{PreprocessingVariationalEquation} the \file{variational equations}{variationalEquation}
comprising the integrated orbit together with the state transition matrix are stored in
\configFile{outputfileVariational}{variationalEquation}.

This program has to be executed for both GRACE or GRACE-FO satellites and it is recommended
to use \program{LoopPrograms}.
\begin{itemize}
    \item \configFile{inputfileSatelliteModel}{satelliteModel}: satellite model from \verb|020groopsInstruments.xml|
    \item \configFile{inputfileOrbit}{instrument}: the approximate orbits from \verb|020groopsInstruments.xml|
    \item \configFile{inputfileStarCamera}{instrument}: the attitude file from \verb|020groopsInstruments.xml|
    \item \configFile{inputfileAccelerometer}{instrument}: the accelerometer data from
    \verb|020groopsInstruments.xml|
    \item \config{forces}: see below
    \item \configClass{ephemerides}{ephemeridesType}: JPL DE432
    \item \configClass{gradientfield:potentialCoefficients}{gravityfieldType:potentialCoefficients}: a static gravity field (GOCO06s) with \config{maxDegree}=\verb|10| is more than sufficient.
\end{itemize}

The \configClass{force models}{forcesType} include:
\begin{itemize}
  \item \configClass{gravityfield:timeSplines}{gravityfieldType:timeSplines}:
        the previously estimated time-variable gravity field
  \item \configClass{tides:astronomicalTide}{tidesType:astronomicalTide}: astronomical tides (based on JPL DE432 ephemerides)
  \item \configClass{tides:earthTide}{tidesType:earthTide}: Earth tide (IERS conventions)
  \item \configClass{miscAccelerations:relativisticEffect}{miscAccelerationsType:relativisticEffect}:
        relativistic effects (IERS conventions)
\end{itemize}

In \program{PreprocessingVariationalEquationOrbitFit} the integrated orbit
(\configFile{inputfileVariational}{variationalEquation}) is fitted to the kinematic orbit
(\configFile{inputfileOrbit}{instrument}) by least squares adjustment. The additional accelerometer
calibration parameters can be defined by
\begin{itemize}
  \item \configClass{parametrizationAcceleration}{parametrizationAccelerationType}: accelerometer scale factor (once per day)
  \item \configClass{parametrizationAcceleration}{parametrizationAccelerationType}: accelerometer bias (time spline with 6h nodes)
\end{itemize}

The observation equations (parameter sensitivity matrix) are computed by integration of the variational
equations (\configFile{inputfileVariational}{variationalEquation}) using a polynomial with
\config{integrationDegree}=\verb|7|. \program{PreprocessingVariationalEquationOrbitFit} has to be
executed per satellite.

\program{PreprocessingVariationalEquationSstFit} fits two dynamic orbits
\configFile{inputfileVariational1/2}{variationalEquation} to the SST observations and the kinematic orbits.
\begin{itemize}
    \item \config{rightHandSide}: input for observation vectors
        \begin{itemize}
            \item \configFile{inputfileSatelliteTracking}{instrument}: K-band range rate observations
            \item \configFile{inputfileSatelliteTracking}{instrument}: light time correction
            \item \configFile{inputfileSatelliteTracking}{instrument}: antenna offset corrections
            \item \configFile{inputfileOrbit1}{instrument}: kinematic orbit of satellite 1
            \item \configFile{inputfileOrbit2}{instrument}: kinematic orbit of satellite 2
        \end{itemize}
    \item \config{sstType}: rangeRate
    \item \configFile{inputfileVariational1}{variationalEquation}: dynamic orbit and integrated state matrix of satellite 1
    \item \configFile{inputfileVariational2}{variationalEquation}: dynamic orbit and integrated state matrix of satellite 2
    \item \configClass{parametrizationAcceleration1}{parametrizationAccelerationType}: same as in In \program{PreprocessingVariationalEquationOrbitFit}
    \item \configClass{parametrizationAcceleration2}{parametrizationAccelerationType}: same as in In \program{PreprocessingVariationalEquationOrbitFit}
    \item \config{integrationDegree}: \verb|7|
    \item \config{interpolationDegree}: \verb|7|
    \item \configClass{covarianceSst}{covarianceSstType}
        \begin{itemize}
            \item \config{sigma}: \verb|1|
        \end{itemize}
    \item \configClass{covariancePod1}{covariancePodType}
        \begin{itemize}
            \item \config{sigma}: 1
            \item \configFile{inputfileCovariancePodEpoch}{instrument}: 3x3 epoch covariances
        \end{itemize}
    \item \configClass{covariancePod2}{covariancePodType}
        \begin{itemize}
            \item \config{sigma}: 1
            \item \configFile{inputfileCovariancePodEpoch}{instrument}: 3x3 epoch covariances
        \end{itemize}
\end{itemize}
The estimated accelerometer calibration parameters from \program{PreprocessingVariationalEquationOrbitFit}
and \program{PreprocessingVariationalEquationSstFit} are determined as corrections and stored
in \configFile{outputfileSolution}{matrix}. Both correction estimates have to be summed up using
\program{FunctionsCalculate}.

The dynamic orbit and the resulting accelerometer calibration parameters are now used to re-integrate
the orbit once more using \program{PreprocessingVariationalEquation} and introducing
\configClass{parametrizationAcceleration}{parametrizationAccelerationType} as \config{estimatedParameters}.
This step usually ensures convergence. If the maximum orbit difference is still not sufficient this step
can be repeated again.

\subsection{Preprocessing}\label{cookbook.gravityFieldGrace:preprocessing}
The script \verb|030groopsPreprocessing.xml| implements the following steps and settings.
The program \program{PreprocessingSst} processes SST observations and kinematic orbit data and performs
a complete least squares adjustment for gravity field determination by computing the observations equations.
Force model parameters (gravitational potential coefficients and accelerometer calibration parameters)
are computed by integrating the parameter sensitivity matrix from the variational equations.
Parameters describing effects due to the SST observation system and geometry (KBR antenna phase
center variations) are computed using the dynamic orbits as a Taylor point.
Short time gravity variations can be co-estimated together with the monthly mean gravity field.
The autoregressive model sequence constraining the short time parameters is provided in the data folder.
See \href{https://doi.org/10.1007/s00190-019-01314-1}{Kvas 2019} for more information about
this co-estimation.
\begin{itemize}
    \item \configClass{observation}{observationType}: sstVariational
        \begin{itemize}
            \item \config{rightHandSide}:
            \begin{itemize}
                \item \configFile{inputfileSatelliteTracking}{instrument}: KBR range rates
                \item \configFile{inputfileSatelliteTracking}{instrument}: light time correction
                \item \configFile{inputfileSatelliteTracking}{instrument}: antenna offset corrections
                \item \configFile{inputfileOrbit1}{instrument}: kinematic orbit of satellite 1
                \item \configFile{inputfileOrbit2}{instrument}: kinematic orbit of satellite 2
            \end{itemize}
            \item \config{sstType}: rangeRate
            \item \configFile{inputfileVariational1}{variationalEquation}: dynamic orbit and integrated state matrix of satellite 1
            \item \configFile{inputfileVariational2}{variationalEquation}: dynamic orbit and integrated state matrix of satellite 2
            \item \configClass{ephemerides}{ephemeridesType}: JPL DE432
            \item \configClass{parametrizationGravity}{parametrizationGravityType}: spherical harmonics from \config{minDegree}=\verb|2| to \config{maxDegree}=\verb|60|
            \item \configClass{parametrizationGravity}{parametrizationGravityType}: high frequency parametrization
            \item \configClass{parametrizationAcceleration1}{parametrizationAccelerationType}: same as in In \program{PreprocessingVariationalEquationOrbitFit}
            \item \configClass{parametrizationAcceleration2}{parametrizationAccelerationType}: same as in In \program{PreprocessingVariationalEquationOrbitFit}
            \item \configClass{parametrizationSst}{parametrizationSatelliteTrackingType}: antenna phase center variations (y and z for both satellites)
            \item \config{integrationDegree}: 7
            \item \config{interpolationDegree}: 7
        \end{itemize}
    \item \configClass{covarianceSst}{covarianceSstType}
        \begin{itemize}
            \item \config{sigma}: \verb|1e-7|
            \item \config{sampling}: 5 [seconds]
        \end{itemize}
    \item \configClass{covariancePod1/2}{covariancePodType}
        \begin{itemize}
            \item \config{sigma}:  \verb|2|
            \item \configFile{inputfileCovariancePodEpoch}{instrument}: 3x3 epoch covariances
            \item \config{sampling}: 60 [seconds]
        \end{itemize}
    \item \config{estimateShortTimeVariations}
        \begin{itemize}
            \item \configClass{autoregressiveModelSequence}{autoregressiveModelSequenceType}: AR models
            \item \configClass{parameterSelection}{parameterSelectorType}: names
                \begin{itemize}
                    \item \configClass{parameterName}{parameterNamesType}: parametrizationGravity
                    \begin{itemize}
                        \item \configClass{parametrization}{parametrizationGravityType}: high frequency parametrization
                    \end{itemize}
                \end{itemize}
        \end{itemize}
\end{itemize}

\program{ParameterSelection2IndexVector} and \program{MatrixCalculate} with
\configClass{matrix:reorder}{matrixGeneratorType:reorder} can be used to extract the desired
spherical harmonic coefficients from \configFile{outputfileSolution}{matrix} and the respective
standard deviations from \configFile{outputfileSigmax}{matrix} up to a certain degree.

In the program \program{Gravityfield2PotentialCoefficients} the estimated spherical harmonics
coefficients are read with
\configClass{gravityfield:fromParametrization}{gravityfieldType:fromParametrization}.
The monthly mean gravity field can be added back by additionaly selecting the time splines created
in \verb|000groopsBackgroundModels.xml| using
\configClass{gravityfield:timeSplines}{gravityfieldType:timeSplines}. The preprocessing solution
is saved as a \file{spherical harmonics file}{potentialCoefficients}.

\subsection{Setting up normal equations}\label{cookbook.gravityFieldGrace:monthlyNormals}
Normal equations are set up in the script \verb|040groopsMonthlyNormals120.xml| using
the program \program{NormalsBuildShortTimeStaticLongTime}. The time intervals which the normal
equations are divided into are defined in \configFile{inputfileArcList}{arcList}.
The normal equations are based on \configClass{observation}{observationType} including the SST data,
the kinematic orbits and the variational equations. The parametrization of the gravity field can
be set with \configClass{observation:parametrizationGravity}{parametrizationGravityType}
(e.g. spherical harmonics up to degree and order 120). Accelerometer calibration parameters
and KBR antenna phase center variations can be parameterized using
\configClass{parametrizationAcceleration}{parametrizationAccelerationType} and
\configClass{parametrizationSst}{parametrizationSatelliteTrackingType}.
With \config{estimateShortTimeVariations} short time variations of the gravity
field can be co-estimated. The parameters selected by
\configClass{parameterSelection}{parameterSelectorType} (e.g. linear splines with 6h nodes) are
constrained by an \configClass{autoregressiveModelSequence}{autoregressiveModelSequenceType}.
Additional temporal variations (e.g. trend and annual oscillation) could be estimated with
\config{estimateLongTimeVariations}.

\subsection{Solving normal equations}\label{cookbook.gravityFieldGrace:monhtlySolve}
The desired spherical harmonic coefficients are determined in the script \verb|050groopsMonthlySolve.xml|.
\program{NormalsSolverVCE} accumulates \configClass{normalEquation}{normalEquationType} and solves
the total combined system. Variance component estimation is used to determine the relative weighting
of the individual normals. The estimated parameter vector (\configFile{outputfileSolution}{matrix})
the estimated accuracies (\configFile{outputfileSigmax}{matrix}) and the full covariance matrix
(\configFile{outputfileCovariance}{matrix}) can be saved.
Using \program{Gravityfield2PotentialCoefficients} the final solution can be saved as
a \file{spherical harmonics file}{potentialCoefficients} by adding back the monthly mean gravity
field to the estimated spherical harmonic coefficients.

% =============================================================
